
\documentclass{article}
\usepackage{amsmath}
\usepackage{graphicx}
\usepackage{hyperref}

\title{Predicting S\textsubscript{11} from Iris Dimensions Using a Neural Network}
\author{Generated by AI based on given input by the user}
\date{\today}

\begin{document}

\maketitle

\section{Introduction}
In this document, we explain the mathematical and scientific reasoning behind predicting the scattering parameter S\textsubscript{11} from iris dimensions using a neural network.

\section{Understanding S\textsubscript{11} and Its Relationship with Iris Dimensions}
\subsection{What is S\textsubscript{11}?}
S\textsubscript{11} (Scattering Parameter, Reflection Coefficient) describes how much of the input signal is reflected back at different frequencies. It depends on the iris dimensions, which act as tunable resonators or filters in the waveguide.

\subsection{Physical Influence of Iris Dimensions}
\begin{itemize}
    \item \textbf{Changing iris widths affects impedance matching:} Wider or narrower irises change how electromagnetic waves propagate through the structure, affecting the amount of reflection at each frequency.
    \item \textbf{Multiple irises introduce multiple resonances:} Different iris shapes and spacings create multiple reflections and transmissions, modifying the frequency response of the waveguide.
\end{itemize}

\section{Mathematical Model: Function Approximation with a Neural Network}
\subsection{General Mapping}
The model learns a function \( f \) such that:

\[
S_{11}(f) = f(\text{iris}_1, \text{iris}_2, \text{iris}_3, \text{iris}_4)
\]

where:
\begin{itemize}
    \item \( S_{11}(f) \) is a vector of 100 values (one for each frequency point).
    \item \( f \) is the complex relationship between iris dimensions and the reflection coefficient.
\end{itemize}

\subsection{Why Use a Neural Network?}
\begin{itemize}
    \item The relationship is \textbf{highly nonlinear} due to the complex wave physics.
    \item Analytical solutions (e.g., Maxwell’s Equations) are too complex to solve directly for arbitrary geometries.
    \item A neural network acts as a \textbf{universal function approximator}, learning the hidden nonlinear transformations between iris dimensions and S\textsubscript{11}.
\end{itemize}

\section{Neural Network Computation}
\subsection{Input Layer}
The model receives iris dimensions as a 4D input vector. These values are normalized before being fed into the model.

\subsection{Hidden Layers}
\begin{itemize}
    \item \textbf{First layer:} 128 neurons, applying a nonlinear ReLU activation.
    \item \textbf{Second layer:} 64 neurons, further refining the feature representation.
    \item \textbf{Final layer:} Outputs a 100-dimensional vector, corresponding to predicted S\textsubscript{11} values over 100 frequency points.
\end{itemize}

Mathematically, this can be expressed as:

\[
S_{11} = NN(\text{iris}_1, \text{iris}_2, \text{iris}_3, \text{iris}_4)
\]

where \( NN \) represents the trained neural network.

\section{Training and Error Minimization}
\subsection{Loss Function}
The model is trained using Mean Squared Error (MSE):

\[
MSE = \frac{1}{n} \sum_{i=1}^{n} (S_{11}^{true} - S_{11}^{pred})^2
\]

\begin{itemize}
    \item \( S_{11}^{true} \) is the actual reflection coefficient from simulations or experiments.
    \item \( S_{11}^{pred} \) is the predicted reflection coefficient.
\end{itemize}

\subsection{Why Include Error in Training?}
\begin{itemize}
    \item The \textbf{"error"} column in your dataset helps the model learn which iris configurations minimize reflection loss.
    \item The model implicitly learns an optimization objective: configurations with lower \textbf{error} should correspond to \textbf{better-matching S\textsubscript{11} curves}.
\end{itemize}

\section{Using the Model for Prediction}
\subsection{Given a New Set of Iris Dimensions}
\begin{enumerate}
    \item \textbf{Normalize} the iris dimensions.
    \item \textbf{Pass through the trained neural network}.
    \item \textbf{Output a 100-dimensional vector}, representing the predicted S\textsubscript{11} curve.
\end{enumerate}

This allows engineers to design waveguide structures by choosing iris dimensions that yield desired reflection characteristics \textbf{without running expensive simulations}.

\section{Conclusion}
This document explained how a neural network maps iris dimensions to S\textsubscript{11} values, considering the underlying wave physics. The model enables rapid prediction of S\textsubscript{11} without the need for computationally expensive simulations.

\end{document}
